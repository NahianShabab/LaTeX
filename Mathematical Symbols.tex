\documentclass[]{article}
\usepackage{amsmath}
\usepackage{amssymb}
\begin{document}
\section{Display Mode Equations}
\subsection{Basic Equation}

\[f(x)=x^2-3x+9\]
\[G(x)=sin^2(x)\]   


\subsection{Alignment of Equations}
\begin{align}
    \nonumber
    f(x) & = x^2-5x+6\\
    & = (x-3)\cdot(x-2)
\end{align}
\begin{align*}
    2x+3y & = 5 & 9x-7y & = 12 & 3x+4y & = 7 \\
    x+9y & = 4  & -2x+5y & = 13 & -x-y & = 1
\end{align*}
\begin{align}
    \nonumber
    F & = mass \times acceleration\\
    & = m \times a
\end{align}
\section{Inline Mode Equations}
The equation of force is given by $F=ma$. This equation was derived by Issac Newton. Here \textit{F} is force, \textit{m} is mass and \textit{a} is acceleration. Again the equation is \(F=ma\). Remember this equation!
\section{Symbols}
\subsection{Basic Arithmetic}

\begin{align*}
6+4\\
6-4\\
6 \times 4\\
6 \cdot 4\\
6 \div 4\\
\dfrac{6}{4}\\
\tfrac{6}{4}
\end{align*}
\newpage
\section{Superscript and Subscripts}
A Polynomial of Degree $n$ is given by,
\begin{align}
f(x)=a_n x^n+a_{n-1}x^{n-1}+a_{n-2}x^{n-2}+\dots+a_1x^1+a_0
\end{align}\\
The equation of a dying signal is given by,
\begin{align}
    S(t)=Ae^{-kt}
\end{align}
\begin{align}
    t_{p_x}^2
\end{align}
\begin{align}
    \sum_{i=0}^{n}i=\frac{n\cdot\left(n+1\right)}{2}
\end{align}
\section{Parenthesis}
\begin{align*}
\left( a+b\right)
\end{align*}
Force is given by,
\begin{align*}
    F=ma \text{ where \(m\) is mass and \(a\) is acceleration }
\end{align*}
\section{Greek Symbols}
alpha $\alpha$, beta $\beta$, gamma $\gamma$, theta $\theta$, omega $\omega$\\
Big Gamma $\Gamma$, Big Delta $\Delta$, Big Theta $\Theta$, Big Omega $\Omega$\\
Insertion sort's time complexity is $O(n^2)$\\
Gamma Function of $n$ is given by $\Gamma\left(n+1\right)=n\cdot\Gamma\left
(n\right)$ and $\Gamma\left(\frac{1}{2}\right)=\sqrt{\pi}$
\section{Recreating Some Famous Equations}
Speed of Light
\begin{align}
    c=\frac{1}{\sqrt{\epsilon_0\mu_0}}
\end{align}
\begin{align}
    T_s=2\pi\sqrt{\frac{m}{k}}
\end{align}
\begin{align}
    \Vec{F_s}=k|\Vec{x}|
\end{align}
\begin{align}
    |\Vec{F_g}|=G\frac{m_1m_2}{r^2}\text{ where $G$ is the gravitational constant }
\end{align}
\begin{align}
    p\left(\theta\right)=\sum_{i=1}^{K}\phi_iN\left(\mu_i,\Sigma_i\right)
\end{align}
\end{document}
